% Options for packages loaded elsewhere
\PassOptionsToPackage{unicode}{hyperref}
\PassOptionsToPackage{hyphens}{url}
%
\documentclass[
]{book}
\usepackage{lmodern}
\usepackage{amssymb,amsmath}
\usepackage{ifxetex,ifluatex}
\ifnum 0\ifxetex 1\fi\ifluatex 1\fi=0 % if pdftex
  \usepackage[T1]{fontenc}
  \usepackage[utf8]{inputenc}
  \usepackage{textcomp} % provide euro and other symbols
\else % if luatex or xetex
  \usepackage{unicode-math}
  \defaultfontfeatures{Scale=MatchLowercase}
  \defaultfontfeatures[\rmfamily]{Ligatures=TeX,Scale=1}
\fi
% Use upquote if available, for straight quotes in verbatim environments
\IfFileExists{upquote.sty}{\usepackage{upquote}}{}
\IfFileExists{microtype.sty}{% use microtype if available
  \usepackage[]{microtype}
  \UseMicrotypeSet[protrusion]{basicmath} % disable protrusion for tt fonts
}{}
\makeatletter
\@ifundefined{KOMAClassName}{% if non-KOMA class
  \IfFileExists{parskip.sty}{%
    \usepackage{parskip}
  }{% else
    \setlength{\parindent}{0pt}
    \setlength{\parskip}{6pt plus 2pt minus 1pt}}
}{% if KOMA class
  \KOMAoptions{parskip=half}}
\makeatother
\usepackage{xcolor}
\IfFileExists{xurl.sty}{\usepackage{xurl}}{} % add URL line breaks if available
\IfFileExists{bookmark.sty}{\usepackage{bookmark}}{\usepackage{hyperref}}
\hypersetup{
  pdftitle={Agricultural Markets},
  pdfauthor={David Ubilava},
  hidelinks,
  pdfcreator={LaTeX via pandoc}}
\urlstyle{same} % disable monospaced font for URLs
\usepackage{longtable,booktabs}
% Correct order of tables after \paragraph or \subparagraph
\usepackage{etoolbox}
\makeatletter
\patchcmd\longtable{\par}{\if@noskipsec\mbox{}\fi\par}{}{}
\makeatother
% Allow footnotes in longtable head/foot
\IfFileExists{footnotehyper.sty}{\usepackage{footnotehyper}}{\usepackage{footnote}}
\makesavenoteenv{longtable}
\usepackage{graphicx,grffile}
\makeatletter
\def\maxwidth{\ifdim\Gin@nat@width>\linewidth\linewidth\else\Gin@nat@width\fi}
\def\maxheight{\ifdim\Gin@nat@height>\textheight\textheight\else\Gin@nat@height\fi}
\makeatother
% Scale images if necessary, so that they will not overflow the page
% margins by default, and it is still possible to overwrite the defaults
% using explicit options in \includegraphics[width, height, ...]{}
\setkeys{Gin}{width=\maxwidth,height=\maxheight,keepaspectratio}
% Set default figure placement to htbp
\makeatletter
\def\fps@figure{htbp}
\makeatother
\setlength{\emergencystretch}{3em} % prevent overfull lines
\providecommand{\tightlist}{%
  \setlength{\itemsep}{0pt}\setlength{\parskip}{0pt}}
\setcounter{secnumdepth}{5}

\title{Agricultural Markets}
\author{David Ubilava}
\date{September 2020}

\begin{document}
\maketitle

{
\setcounter{tocdepth}{1}
\tableofcontents
}
\hypertarget{preamble}{%
\chapter*{Preamble}\label{preamble}}
\addcontentsline{toc}{chapter}{Preamble}

This is a collection of notes designed to teach Agricultural Markets at an undergraduate level. The content (loosely) follows Tomek and Kaiser (\protect\hyperlink{ref-tomek2014}{2014}), and covers most topics highlighted in Myers, Sexton, and Tomek (\protect\hyperlink{ref-myers2010}{2010}).

\hypertarget{preferences-and-demand}{%
\chapter{Preferences and Demand}\label{preferences-and-demand}}

Tomek and Kaiser (\protect\hyperlink{ref-tomek2014}{2014}, Chapters 2 \& 3)

\hypertarget{choice-and-individual-demand}{%
\section{Choice and Individual Demand}\label{choice-and-individual-demand}}

A consumer at any given time faces an array of choices. These choices are linked with their \emph{needs}. Ideally, the consumer would like to satisfy all of these needs, but their choices are constrained by their disposable income---the budget allocated for purchasing goods. The objective of a consumer, then, is to select a subset of goods that \emph{best} satisfy their needs, subject to their budget constraint.

Economists define ``best'' in terms of consumers' attempts to maximize their \emph{utility}. Utility is an economic term describing satisfaction from the consumption of goods. While it is a unit-free measure, in general, more utility is better than less utility, and positive utility is ``good'' and negative utility is ``bad.''

Mathematically, for a set (or a basket) of goods, given by \(\mathbf{x}=(x_1,\ldots,x_n)'\), the utility approach to the theory of demand is given by: \[\max U(\mathbf{x}),\;~~s.t.\;~\mathbf{p}'\mathbf{x} \leq M,\] where \(\mathbf{p}=(p_1,\ldots,p_n)'\) is a vector of prices for the given set of goods, and \(M\) is the total disposable income (i.e., the budget). Thus, a consumer tries to attain the maximum utility they can afford, given their income and the prices of goods. The following graph illustrates this for a scenario with two goods.

\begin{figure}

{\centering \includegraphics[width=0.8\linewidth]{agmarkets_files/figure-latex/demand-1} 

}

\caption{Individual Choice and Demand}\label{fig:demand}
\end{figure}

Market determines prices of goods---i.e., a consumer has no control over the prices. Changes to prices affect a consumer's decision--making. A consumer can afford and, typically, will purchase more of the same good as the price of the good drops. This is known as the \emph{substitution effect}. The substitution effect is always inversely related to the price change.

Moreover, a decrease in price of a good, in effect, is equivalent to an increase in income---a consumer can afford more of all goods with the same amount of money. This is known as the \emph{income effect}. The income effect, usually (but not necessarily), is inversely related to the price change.

Thus, a price drop will typically increase the quantity demanded due to the income effect as well as the substitution effect.

For some goods, real income and demand are inversely related. This is equivalent to the negative income effect due to price decrease. Such goods are called \emph{inferior goods}. For inferior goods, the income and the substitution effects have opposite signs.

At the extreme, the income effect might outweigh the substitution effect, resulting in a positively sloped demand curve. This is known as \emph{Giffen's paradox}, and such goods are referred to as the \emph{Giffen goods}. It might occur, for example, when a staple commodity, such as rice, constitutes a large portion of consumer's expenditures.

\hypertarget{market-demand}{%
\section{Market Demand}\label{market-demand}}

We obtain aggregate or market demand when we add up (horizontally) individual consumer demand functions. Market demand is a schedule of product quantities that all consumers in a market are willing to purchase at given prices, everything else held constant. Normally, the demand curve is downward sloping, which is inferred in the \emph{law of demand}.

Own price of the good is a sole factor affecting the quantity demanded, which results in \emph{movement along the demand curve}. All other factors may influence demand by \emph{shifting the demand curve}. These factors can be economic (e.g., prices and availability of other goods and services, income and its distribution); demographic (population size and its distribution by age, gender, ethnicity, etc.); tastes and preferences (influenced by information and advertising, lifestyle, etc.).

\hypertarget{demand-elasticities}{%
\subsection{Demand Elasticities}\label{demand-elasticities}}

The responsiveness of quantity demanded to a price change---known as the own--price elasticity of demand (demand elasticity)---may vary from product to product, over time, and across locations. For example, quantity of salt purchased will not vary greatly due to the price change, but quantity of bananas purchased will likely be more responsive to price changes. From the consumers' standpoint: bananas have substitutes, but salt doesn't. As another example, quantity of rice purchased is likely to be less responsive to price changes in Asian countries, where it is considered a staple food, than in the Western world, where rice is viewed more as a side dish, perhaps with more substitutes.

Mathematically, price elasticity is given by: \[\epsilon = \frac{\partial Q}{Q}/\frac{\partial P}{P},\] where \(\partial\) denotes an infinitesimal change. Note that \(\frac{\partial X}{X} \equiv \%\Delta X\), and so price elasticity is defined as the percentage change in quantity relative to the percentage change in price. That the elasticity is measured in relative (or percentage) terms, makes it an attractive measure as it facilitates a direct comparison between goods with possibly different units of measurement of quantities and prices.

The foregoing equation can be rewritten as: \[\epsilon = \frac{\partial Q}{\partial P}\frac{P}{Q},\] where \(\frac{\partial Q}{\partial P}\) is the slope of the demand curve. Because of the downward-sloping demand curve, the price elasticity measure ranges from zero to negative infinity. This range consists of \emph{inelastic} (\(|\epsilon_{ii}| < 1\)) and \emph{elastic} (\(|\epsilon_{ii}| > 1\)) segments, with \(|\epsilon_{ii}| = 1\) representing the \emph{unitary elastic} case.

For most functional forms, including the linear demand, the elasticity coefficient varies along the demand curve. Exceptions are if demand is represented by a straight horizontal line, a straight vertical line, a power function, and a rectangular hyperbola. In general, however, demand for different goods is usually categorized as elastic or inelastic.

The responsiveness of the demand for a good \(i\) to a price changes of a related good \(j\) is known as the cross--price elasticity of demand (cross--price elasticity). Mathematically, cross-price elasticity is given by: \[\epsilon_{ij} = \frac{\partial Q_i}{Q_i}/\frac{\partial P_j}{P_j},\] or, alternatively: \[\epsilon_{ij} = \frac{\partial Q_i}{\partial P_j}\frac{P_j}{Q_i}\]

Three types of relationships---which are related to the substitution effect---can be identified depending on the sign of cross-price elasticity: for substitute goods, \(\epsilon_{ij} > 0\); for complement goods, \(\epsilon_{ij} < 0\); and for independent goods, \(\epsilon_{ij} = 0\)

Note that the price change also facilitates the income effect. If the income effect outweighs the substitution effect, we might observe a negative relationship between the demand for product \(i\) and the change in price of product \(j\), even when the two are substitutes. To the extent that the expenditure share of a given good is a small fraction of the disposable income, the substitution effect will typically dominate the income effect.

The responsiveness of the demand for a good to income changes is income elasticity of demand (income elasticity). Mathematically, income elasticity is given by: \[\epsilon_{m} = \frac{\partial Q}{Q}/\frac{\partial M}{M},\] or, alternatively: \[\epsilon_{m} = \frac{\partial Q}{\partial M}\frac{M}{Q}\]

In most cases, the income elasticity of demand has a positive value: as income increases, consumers tend to buy more of most products. However, income elasticities tend to decline as incomes increase. In general, demand is thought to be a nonlinear function of income. In practice, expenditures (rather than income) are often used to estimate ``income'' elasticities.

\hypertarget{production-and-supply}{%
\chapter{Production and Supply}\label{production-and-supply}}

Tomek and Kaiser (\protect\hyperlink{ref-tomek2014}{2014}, Chapters 4 \& 5)

\hypertarget{costs-and-supply-by-a-single-firm}{%
\section{Costs and Supply by a Single Firm}\label{costs-and-supply-by-a-single-firm}}

Much like consumers are utility--maximizers, the firms are profit--maximizers. For a firm producing a product, \(q\), using a set of inputs, \(z=(z_1,\ldots,z_k)'\) , the short-run profit is defined as: \[\pi = pq-w'z,\] where \(p\) is a unit price of \(q\), and \(w=(w_1,\ldots,w_k)\) is a vector input costs.

A firm's output is related to the inputs through a production function: \[q = q(z)\]

The profit function, thus, can be rewritten as: \[\pi = pq(z)-w'z.\] This equation can be used in deriving the producer's supply function of output and demand functions for inputs.

A producer maximizes profits by using each factor up to the point where the last unit just pays for itself. Optimal use of input is determined with respect to the (known) production function. The functional form of production is such that output is increasing with input, but at a decreasing rate. The rate of increase in output is called the \emph{marginal (physical) product} of input: \[MP_i=\frac{\partial q}{\partial z_i}.\]

Following the rule of profit maximization, the optimum level of factor use is at the point where the \(MP_i\) multiplied by price of output, \(p\), just equals to the unit cost of input, \(w_i\), which yields the following relationship:\[\frac{\partial q}{\partial z_i} = \frac{w_i}{p}\]

This relationship implies that an increase of the input cost, will lead to reduced use of that input and lower output, everything else held constant. Moreover, optimal factor use will change only with an increase or decrease in relative prices.

\hypertarget{market-supply}{%
\section{Market Supply}\label{market-supply}}

Market supply is a (horizontal) aggregate of individual firm's supply curves. Assuming \(n\) identical firms, each producing \(q\) units of optimal output, the aggregate market supply is \(Q = nq\). The supply curve, generally, is upward sloping, which is inferred in the \emph{law of supply}.

Own price of a good is a sole factor affecting the quantity supplied resulting in the movement along the supply curve. All other factors affect supply by \emph{shifting the supply curve}. Such factors can be: prices of inputs; prices of related goods; goods competing for the same factors of production (e.g., the same land for corn and soybeans); joint production of goods (e.g., soybean meal and soybean oil); factors affecting production costs and output quantities (e.g., technology and weather); production risks; taxes and subsidies, etc.

\hypertarget{supply-elasticity}{%
\subsection{Supply Elasticity}\label{supply-elasticity}}

The responsiveness of the supply to price changes is known as \emph{price elasticity of supply} (supply elasticity). Mathematically, supply elasticity is given by: \[\eta = \frac{\partial Q}{Q}/\frac{\partial P}{P} \equiv \frac{\partial Q}{\partial P}\frac{P}{Q}\]

In the very short run, supply is perfectly inelastic (vertical line) and the elasticity is zero. Inelastic supply refers to elasticity measures between \(0\) and \(1\). Elastic supply refers to elasticity measures greater than \(1\). A perfectly elastic supply (horizontal line) has the elasticity measure or \(\infty\).

Assuming a linear supply curve, as quantity and price increase (i.e., as we move upwards along the supply curve) the price elasticity of supply converges to \(1\). Moreover, if a linear supply curve intersects with the origin, the elasticity of supply is always \(1\).

\hypertarget{market-structure}{%
\section{Market Structure}\label{market-structure}}

\hypertarget{perfect-competition}{%
\subsection{Perfect Competition}\label{perfect-competition}}

To begin, we assume that there are many sellers and buyers in a market, and that each individual seller or buyer cannot influence the market. Furthermore, we assume that the product is homogenous, that there are no costs associated with entering and leaving the market, and that the information on economic forces that are determining prices is complete and freely available for market participants. Such market is referred to as a perfectly competitive market.

A perfectly competitive market is efficient, in that the market equilibrium leads to the socially optimal outcome. Such markets only exist in theory, however, as one or several assumptions of a perfectly competitive market are usually violated.

\hypertarget{monopoly}{%
\subsection{Monopoly}\label{monopoly}}

On the other end of the spectrum, we have a monopoly (or a monopsony). In the case of a monopoly, we have a single seller (in the case of a monopsony we have a single buyer). The product homogeneity assumption is automatically met. However, there are very high costs associated with entering the market.

A monopolist, unlike a competitive firm, is no longer a price-taker; instead, they can `manipulate' quantity demanded by changing prices, or (equivalently) they can manipulate market prices by changing their output. Market equilibrium under monopoly leads to an inefficient outcome, as some of the surplus - which would have been attained under a perfectly competitive market - is lost; this is known as the \emph{dead-weight loss}.

\hypertarget{oligopoly}{%
\subsection{Oligopoly}\label{oligopoly}}

In-between of a perfectly competitive market and a monopoly (or a monopsony) we have oligopoly (or oligopsony), which means several sellers (or buyers) on the market.

If these sellers collude, the market outcome under oligopoly can be identical to that under monopoly. Even if they do not collude, but act strategically, they can still extract additional surplus from consumers, also leading to an inefficient outcome and the dead-weight loss (albeit not as large as that under monopoly). In some instances - e.g., if oligopolists engage in a `price war' - the resulted market equilibrium can be similar to that under a perfectly competitive market.

\hypertarget{marketing-system-and-margins}{%
\chapter{Marketing System and Margins}\label{marketing-system-and-margins}}

Tomek and Kaiser (\protect\hyperlink{ref-tomek2014}{2014}, Chapters 6 \& 10); Bellemare and Bloem (\protect\hyperlink{ref-bellemare2018}{2018})

When we enjoy our pizza at a restaurant, or an oven roasted chicken at home, rarely (if at all) we contemplate about the chain of events---a system, and the actions of people involved in this system---that has brought the meal to our table. This \emph{agricultural marketing system}, also known as the \emph{food system}, links producers with consumers by transforming an agricultural commodity into a food product.

\hypertarget{vertical-coordination}{%
\section{Vertical Coordination}\label{vertical-coordination}}

Although farmers and consumers sometimes directly interact (e.g., farmers markets), most food products usually go through a complex processing and distribution system after they leave the farm-gate and before they land on a shelf of a retail store. In the process, the usually bulky, perishable, and homogeneous farm commodities are transformed into more concentrated, storable, and differentiated food products.

The agricultural marketing system consists of multiple stages, and involves a number of intermediaries, such as processors, distributors, wholesalers, and retailers. Between each stage of the system there is a market. Within each market a price is determined. Prices aggregate a great deal of information, and play an integral role in conveying this information between food consumers and commodity producers.

The longer and more complex is the marketing system, the more difficult it becomes to relay information from a consumer to a farmer. A profit-maximizing farmer will only supply a better quality commodity, or add a certain attribute to the commodity, if they are rewarded for it. When the reward exceeds costs, the farmer will take actions to supply a commodity with desired qualities. In presence of uncertainty surrounding the rewards (prices), as well as information asymmetry among buyers and sellers within the marketing system, ensuring that the desired product attributes are present for the right price may be a challenging (if not an impossible) task.

One way to facilitate a coordinated performance among members of the supply chain is by means of contracting. A contract is an arrangement between a buyer and a seller, where the parties agree on quantity and price of a commodity to be exchanged months in advance. There are different degrees of `involvement' in contractual agreements; in more `involved' instances, a processor supplies all the inputs, and assumes market risks, while a farmer supplies farm facilities and is responsible for agreed-upon commodity production.

At the extreme, two or more segments of a supply chain go under the same ownership, which is referred to as \emph{vertical integration}. Downstream integration is when a firm begins producing inputs that they previously purchased from their supplier. Upstream integration is when a firm begins performing a function of a firm that previously purchased their product.

Among the key benefits of vertical integration are:

\begin{itemize}
\tightlist
\item
  lower transaction costs;
\item
  improved risk management;
\item
  increased efficiency in resource allocation, production, and distribution.
\end{itemize}

\hypertarget{marketing-margins}{%
\section{Marketing Margins}\label{marketing-margins}}

In the farm-retail price relationship, the difference between what consumers pay and what farmers receive is called \emph{price spread}, also known as \emph{marketing margin}. Consider a profit maximizing retailer, which uses a farm commodity and marketing services as inputs in production. The profit function of this retailer can be given by: \(\pi = p^r q^r - p^f q^f - c^r(q^r)\) where \(p^r\) and \(p^f\) are retail and farm prices; \(q^r\) and \(q^f\) are product output and commodity input quantities; and the last term, \(c^r(q^r)\), represents the retailer's marketing cost function.

For simplicity, let \(q^r=q^f=q\), which reduces the profit function to: \(\pi = (p^r - p^f) q - c^r(q)\), where \(p^r - p^f\) is the price spread.

Marketing costs, \(c^r(q)\) can be modeled as a linear function of quantity produced: \[C^r(q) = \delta^r q,\] where \(\delta^r > 0\) is a constant. In such case, marginal marketing cost is the same for any \(q\): \[\frac{\partial c^r(q)}{\partial q} = \delta^r\;~~\forall\;q.\] This model is consistent with a simple markup rule: the retail price is equal to the wholesale price plus a constant (which is based on the cost of the marketing inputs).

Alternatively, and perhaps more realistically, marketing costs can be modeled, as nonlinear function of quantity produced. The usual assumption, in that case, would be that the first derivative of the marketing costs function should be positive, while the second derivative can be positive or negative (i.e., marketing costs should be increasing with output, but they may be increasing at an increasing rate or a decreasing rate).

Because of a complex marketing system, variables other than just prices and quantities of inputs and outputs may be important in determining the margins. Some of such factors include changes in cost of the marketing services that get the farm product to the consumer in the form they demand (e.g., improvements in technology in the marketing system); changes in factor composition, over time, that may increase the total cost of marketing services (e.g., additional marketing services can emerge, that were not available or feasible before); changes in government programs (e.g., taxes and subsidies); changes in market power of firms in the marketing system, resulting in: increased prices paid by a consumer, or reduced prices received by a farmer (or both).

\hypertarget{elasticity-of-derived-demand}{%
\section{Elasticity of Derived Demand}\label{elasticity-of-derived-demand}}

Assuming linear demand functions, and constant marketing costs, elasticity of derived demand (faced by a farmer) can be given by: \(\epsilon^f = \epsilon^r\frac{p^f}{p^r}\) where \(\epsilon^r\) is elasticity of primary demand (in a retail market), and \(p^f\) and \(p^r\) are the farm commodity and the retail product prices. It follows that the farm-level demand is more price inelastic than the retail demand. The larger is the marketing margin the larger is the difference between the two elasticities to the point that even though the retail demand is price elastic, the derived demand may end up being price inelastic.

\hypertarget{spatial-price-relationships}{%
\chapter{Spatial Price Relationships}\label{spatial-price-relationships}}

Tomek and Kaiser (\protect\hyperlink{ref-tomek2014}{2014}, Chapter 8), Barrett and Li (\protect\hyperlink{ref-barrett2002}{2002})

\hypertarget{spatial-arbitrage-and-market-integration}{%
\section{Spatial Arbitrage and Market Integration}\label{spatial-arbitrage-and-market-integration}}

Agricultural commodities typically are produced in geographically diverse locations, and can be costly to transport. Relationships among geographically distant, or spatially separated markets are often examined by comparing prices in these markets (which, sometimes, may be the only trade data available).

To the extent that supply and demand forces differ across markets, prices of agricultural commodities can (and do) vary spatially. Given a competitive market structure, spatial price relationships are determined by \emph{transfer costs} among markets. In addition to the transportation costs, components of the transfer cost also include other transaction costs related to contracting, insurance, financing, etc.

Spatial arbitrage, through the actions of economic agents, ensures that the price differential between regions \(i\) and \(j\), \(p_{jt}-p_{it}\) at any given time, \(t\), does not exceed the transfer costs, (\(\tau_{ijt}\)): \[p_{jt}-p_{it} \le \tau_{ijt}.\] This is referred to as \emph{long--run competitive equilibrium}. The condition will hold as an equality if the two regions engage in a trade with each other, thereby exhausting the rents to spatial arbitrage. That is, when trade occurs, price differentials should move one-for-one with the costs of spatial arbitrage; but when no trade occurs, there may be no correlation among market prices even though competitive equilibrium holds.

Market integration, defined as tradability or contestability between markets, is a (related but) slightly different concept from competitive spatial equilibrium. This implies the transfer of excess demand from one market to another, manifest
in the physical flow of commodity, the transmission of price shocks from one market to another, or both.

\emph{Spatial equilibrium models} can help determine the optimum trading patterns, given the supply and demand conditions within each country (market). We illustrate this using two competitive markets.

\begin{figure}

{\centering \includegraphics[width=0.8\linewidth]{agmarkets_files/figure-latex/trade-1} 

}

\caption{Spatial Equilibrium}\label{fig:trade}
\end{figure}

\hypertarget{the-law-of-one-price}{%
\section{The Law of One Price}\label{the-law-of-one-price}}

From the foregoing it follows that two spatially separated markets can be in equilibrium but not necessarily integrated (e.g., when the spatial arbitrage condition is satisfied but no trade occurs); they can also be integrated but not necessarily in equilibrium (e.g., when trade occurs but the rents to spatial arbitrage are not exhausted). When the integrated markets are in equilibrium, the \emph{law of one price} (LOP) holds. The LOP suggests that, abstracting from transfer costs, regional markets that are linked by trade and arbitrage, will have a common, or unique price. While the LOP (in its strong form) rarely holds in practice, there is evidence for LOP among studies that explicitly account for transactions costs, as well as those that explicitly focus on traded rather than non-traded commodities.

Notably, price co-movement between two regions can arise for reasons other than those that link these regions via trade. On the other hand, prices that satisfy the LOP may not move together, if transfer costs are large and volatile. Finally, it is not necessary for two regions to be direct trading partners for an integration to be present. As long as these regions are part of a common trading network, they may be integrated just as strongly as if they were direct trading partners.

Several important implications follow. First, because trade will only occur when the price differential equals or exceeds the transfer cost, shocks to production or consumption may result in discontinious trade flows over time. Second, to the extent that trade takes place, the price differential will remain the same (and be equal to the transfer cost). Finally, integrated markets will be characterized by the simultaneous determination of prices, trade, and storage.

Key to the LOP is the notion of \emph{efficient markets}. In general, efficiency is meant to imply that the allocation of resources is such that aggregate welfare cannot be further improved by reallocating of resources. In the context of spatial arbitrage, market efficiency is interpreted to imply that no opportunities for profits have been left unexploited by arbitrageurs. That is, the efficient market hypothesis assumes away profitable arbitrage opportunities. To that end, the LOP may not necessarily apply if buyers have less than perfect information about where to find the lowest price commodity; nor may they apply across international borders, because some goods are not transferable. These principles facilitate the so-called \emph{structure of prices}---a function of the pattern of trade and transfer costs between regions that engage in trade.

Implications of the structure of prices are that: the lowest-cost source determines the price prevailing in each deficit market; producers sell in whichever market yields the highest net return; the price prevailing in each surplus producing area is the deficit market price less the cost of transferring a product to that market. Of course, the foregoing assumes a competitive market structure, with a homogeneous commodity, informed traders, and no barriers to trade.

\hypertarget{international-trade-and-restrictions}{%
\chapter{International Trade and Restrictions}\label{international-trade-and-restrictions}}

Grant and Boys (\protect\hyperlink{ref-grant2012}{2012})

Countries trade because some can produce certain commodities when others cannot (absolute advantage), or, more typically, because some can produce certain commodities more efficiently than others (comparative advantage).

In absence of transfer costs, the intersection of the excess demand and excess supply curves yields the world price. A more realistic scenario is trade with transfer costs. A way to think of the effect of transfer costs is by incorporating them into the supply function of the exporting country.

Transfer costs increase the price at which commodities are traded, and reduce the quantity of traded commodities. As a result, in an importing market, consumers receive less of the commodity for a higher price, and producers sell more of the commodity for a higher price; the opposite is true in an exporting market.

\hypertarget{gains-from-trade}{%
\section{Gains from Trade}\label{gains-from-trade}}

Gains from trade can be grouped into two categories: gains from exchange, and gains from specialization.

Drawing supplies from a world market allows access to a wider array of products at lower cost and perhaps with greater security of supply than can be produced by domestic industries. Lower prices from imported goods allow consumers to buy more goods from disposable income. Lower prices for imported raw materials, that are used to produce final goods, also benefit consumers by lowering prices of these final goods. All these facilitate greater savings, which lead to increased investment and greater economic growth.

Trade stimulates the expansion of low cost industries (and forces the contraction of high cost industries). Increase of the size of the market allows firms or industry to take advantage of economies of scale. Increase of competition provides greater emphasis on technological development and innovation, and results in increased skills of workforce.

\hypertarget{trade-liberalization-vs-protectionism}{%
\section{Trade Liberalization vs Protectionism}\label{trade-liberalization-vs-protectionism}}

The case for free trade is that producers and consumers allocate resources most efficiently when governments do not distort market prices through trade policy. Economic welfare of a small country is highest with free trade. With restricted trade: consumers pay higher prices (than they would have paid otherwise); distorted prices cause overproduction either by existing firms producing more or by more firms entering the market.

Protectionism consists of set of actions that deviates the trade patterns from what would have been their state in absence of any regulations (i.e., in a free trade regime). The case for protectionism is primarily directed to maintain a considerable degree of self--sufficiency in the country, but can also be `re-active' to ongoing economic or political processes. Common levers for trade restriction are tariffs and quotas. Other government interventions that can play the role of trade distortions are taxes and subsidies.

In agriculture, strongly divergent climatic conditions around the globe, create ideal conditions for making use of comparative advantage. And while it would seem that free trade should be particularly attractive and relevant in agriculture, it has been traditionally an area where governments have been reluctant to open up domestic markets to international trade. Typical pattern is that developed countries tend to maintain high domestic prices, to protect farmers' incomes, while developing countries tend to keep domestic prices low, in the interest of consumers.

\hypertarget{trade-negotiations-and-gatt}{%
\section{Trade Negotiations and GATT}\label{trade-negotiations-and-gatt}}

After the World War II, nations around the world recognized the urgent need for coordinated policy efforts in the areas of economic development, monetary policy, and international trade. With that in mind---and after the United States Congress refused to sign on an agreement reached by the Bretton Woods participants to form the International Trade Organization---23 nations signed a multilateral agreement regulating international trade known as the General Agreement on Tariffs and Trade (GATT) in Geneva on 30 October 1947 to take effect on 1 January 1948. GATT became the vehicle for trade negotiations for the next half century, until its successor---the World Trade Organization (WTO) was formed and took effect in 1995.

Main principles of the GATT involved:

\begin{itemize}
\tightlist
\item
  \emph{most--favoured--nation} principle, meaning that trade benefits conferred on one country should be extended unconditionally to all other member countries;
\item
  \emph{national treatment}, meaning that imports should be treated no less favourably than domestic products; and
\item
  \emph{reciprocity}, meaning that negotiations should be conducted on a reciprocal and mutually advantageous basis.
\end{itemize}

\hypertarget{agriculture-in-trade-negotiations}{%
\subsection{Agriculture in Trade Negotiations}\label{agriculture-in-trade-negotiations}}

Idea of replacing agricultural price support with direct payments to farmers decoupled from production dates back to the late 1950s. Twelfth session of the GATT Contracting Parties selected a Panel of Experts, chaired by Gottfried Haberler, to examine the effect of agricultural protectionism, fluctuating commodity prices, and the failure of export earnings to keep pace with import demand in developing countries. The report (1958) found that there was a decline in the terms of trade for primary commodities (relative to the manufactured goods).

This finding accords with what later became to be known as the \emph{Prebisch--Singer hypothesis} (though Haberler himself had particular disagreement was with the idea that there was a systematic long-term decline in the terms of trade). The report stressed the importance of minimizing the effect of agriculture subsidies on competitiveness; recommended replacing price support with direct supplementary payments not linked with production; anticipated discussion on green box subsidies---the kind of subsidies that do not distort trade, or at most cause minimal distortion; that are government-funded (not by charging consumers higher prices); and that do not involve price support.

In 1980s, global economy had entered a cycle of recession, which led to a collapse of agricultural commodity prices. Financial and economic costs of agricultural policies rose sharply in many industrialized countries. Perception that opening up markets could improve economic conditions, and generate much needed efficiency gains, facilitated a new round of multilateral trade negotiations.

In leading up to the 1986 negotiations, developed countries strongly resisted compromises on agriculture. This opposition was subsequently neutralized by guaranteeing farmers continued support, and allowing subsidies that cause not more than minimal trade distortion in order to meet public policy aims.

\hypertarget{uruguay-round}{%
\subsection{Uruguay Round}\label{uruguay-round}}

Agreement on Agriculture, which was reached in the third and final phase of the Uruguay Round, continues to be the most substantial trade liberalization agreement in agricultural products in the history of trade negotiations. The agreement revolved around the following three directions:

\begin{itemize}
\tightlist
\item
  Market access: convert of all existing non-tariffs into tariff equivalents (and, where necessary, guarantee minimum access to domestic markets through the creation of tariff-rate quotas); developed countries to reduce tariffs by 36\% over six years, with a minimum rate of reduction of 15\% for each tariff lines; developing countries to cut tariffs by an average of 24\% over 10 years; safeguard provisions to allow importers to guard against import surges and low world prices.
\item
  Export competition: export subsidies were capped and then reduced in both value (36\%) and volume (21\%). In Nairobi in 2015, WTO members agreed that developed countries would immediately remove export subsidies except for a handful of agriculture products and that developing countries would do so by 2018.
\item
  Domestic support: reduction of the level of sector--wide (rather than product--specific) support---better known as the aggregate measurement of support---by 20\% over six years for developed countries, and 13\% over 10 years for developing countries.
\end{itemize}

\hypertarget{dynamic-price-relationships}{%
\chapter{Dynamic Price Relationships}\label{dynamic-price-relationships}}

Tomek and Kaiser (\protect\hyperlink{ref-tomek2014}{2014}, Chapter 9), Goodwin, Grennes, and Craig (\protect\hyperlink{ref-goodwin2002}{2002})

Over time, demand and supply are influenced by changes to their underlying factors. The dynamic behaviour of prices is a consequence of such changes. This may result in trends, cycles, and seasonality in the commodity prices dynamics. A commodity price series may exhibit one or several of these features (e.g., seasonality is a likely feature of annually harvested agricultural commodity prices; prices of meat typically resemble cyclical patterns, etc.). Over time, the underlying factors of a commodity price behaviour may change. For example, previously perishable products---such as meat, eggs, butter---became ``storable'' with the invention of refrigeration, which had an important impact on price dynamics of the affected agricultural commodities.

\hypertarget{trends-and-shifts-in-price-levels}{%
\section{Trends and Shifts in Price Levels}\label{trends-and-shifts-in-price-levels}}

Commodity prices may exhibit shifts in average price levels. A continuum of such shifts result in price trends. Trends usually are manifested by persistent upward (or downward) movements in prices over a reasonably long period of time. Recall that observed prices are equilibrium market prices, governed by supply and demand schedules. For an observable trend in a commodity price series, continuous relative shifts in supply and demand must occur. For example, a continuous stream of new production technology may cause the supply function to steadily shift rightward, resulting in a downward trend in observed prices.

For agricultural commodities, a major factor in year-to-year price variability is change in annual supply. Crops have swings in annual production, because changing expectations about returns as well as climatic and biological factors of production. Inventories and imports may mitigate the effects of a small crop in a particular region, unless the effect has worldwide manifestation.

In contrast to year-to-year changes, there are occasions when price levels shift to (and remain at) a new level. This may happen as a result of an abrupt (or, possibly, gradual within a relatively short period of time) structural change in commodity markets. For example, the entry of the former Soviet Union into the world market for grains in 1973 created a new source of demand that persisted, which resulted in a substantially high nominal prices. More recently, the so-called ``ethanol boom'' in the U.S. has shifted the prices of corn to a new, higher level. Although it is difficult to ascertain whether a change in a price level is transitory of persistent, shifts in price levels should be distinguished from continuous trends.

\hypertarget{price-cycles-and-the-cobweb-model}{%
\section{Price Cycles and the Cobweb Model}\label{price-cycles-and-the-cobweb-model}}

A cycle is a pattern that repeats itself over a time period that is longer than one year. The simplest variant of a price cycle has a fixed period (length of time, say, from peak to peak), but it rarely is the case. Cycle-like behaviour of prices is typically initiated by an external event (say, a drought). Such events may manifest themselves irregularly, and thus price cycles may also be characterized with irregular cycles. Moreover, due to the very nature of production process, the cycles may be asymmetric -- a positive shock of a given magnitude may result in subsequent dynamic behaviour, which may be different from dynamics due to a negative shock of the same magnitude.

Two factors facilitate cyclical behaviour in commodity prices: (i) the way expectations are formed, and (ii) the costs associated with responding to changed expectations. Due to the production lag, profit-maximizing decisions rely on expected (rather than actual) prices.

To illustrate the point, consider a case of \emph{naive expectations}: \[p_{t+1}^{*} = p_{t}\] where \(p_{t+1}^{*}\) denotes the expected price for period \(t+1\), which is equal to an observed price in period \(t\). In addition, assume a competitive market (producers are price-takers), a market clearing price adjustment, and static supply and demand. Under the foregoing assumptions,

Somewhat more ``sophisticated'' models assume some form of \emph{adaptive expectations}. Such models incorporate multiple lags of prices (i.e., \(p_{t-1},p_{t-2},\dots\)) in forming the expected price of a commodity. The expected price then can be a geometric-weighted average of current and past prices, or a prediction from an autoregressive process. To that end, the naive expectations is a special case of the adaptive expectations.

\hypertarget{seasonality-of-agricultural-commodity-prices}{%
\section{Seasonality of Agricultural Commodity Prices}\label{seasonality-of-agricultural-commodity-prices}}

Seasonal price behaviour is a systematic pattern that occurs within a year, and repeats across the years. For food and agricultural commodities, the main source of seasonality is the supply-side effects. Although, demand-side effects are also occasionally evident (e.g., high demand for turkey meat in the U.S. around the Thanksgiving period in November). Assuming an annually produced storable commodity, and a perfectly competitive market, prices will be lowest just after the harvest, and rise at the rate of \emph{cost of storage} per unit of time.

The typical seasonal price pattern does not prevail each year. Prices may rise by more than the cost of storage, or they may even decline over the season. Economic agents act upon information related to expected production, available stocks, and expected changes in demand for a commodity. But the information is subject to change throughout the season. And prices will reflect those changes.

The storage costs within a year, implicitly depicted in the seasonal pattern, can be divided into four components: The costs of inputs; The opportunity costs (which depends on the price of the commodity and interest rates); The convenience yield (of holding stocks); The risk associated with the expected future price of a commodity

Not only prices increase, but they also become more variable as the storage season progresses. i.e., the price variability immediately after harvest is smaller than the price variability during months later in the marketing year. Within a marketing year, inventories are declining, and when inventories are small late in the season, changes in expectations can have a large price effects (spikes).

\hypertarget{economics-of-storage}{%
\section{Economics of Storage}\label{economics-of-storage}}

Storage has important price--stabilizing role over the marketing year, as it creates a pricing buffer by shifting stocks forward. When aggregated across the whole market, the storage adjustments by individual traders reduce price fluctuation over time.

In general, the driving force in price behaviour is \emph{intertemporal arbitrage} by economic agents. Mathematically, intertemporal arbitrage is given by the following equilibrium condition: \[E\left(p_{t+1}|\Omega_t\right)-p_{t} = s_t,\] where \(s_t\) is the marginal cost of storage between periods \(t\) and \(t+1\). That is, for any fixed period of storage, difference between the expected future price and the current price of a commodity must exactly equal the marginal cost of storage for that time interval.

The aforementioned is referred to as the law of one price in a temporal context. If, for example, the expected price exceeds the current price by more than the cost of storage, an incentive would exist to store a larger amount for future use. This, in turn, will have the effect of raising the current price and reducing the expected price. The process will continue until the equilibrium condition is met.

\hypertarget{futures-markets}{%
\chapter{Futures Markets}\label{futures-markets}}

\hypertarget{futures-contracts}{%
\section{Futures Contracts}\label{futures-contracts}}

Futures markets provide a venue for trade of \emph{futures contracts}. A futures contract is a legal instrument, enforceable by the rules of the exchange on which it is traded, to deliver or accept delivery of a specific amount of a commodity during a specified month at a price established in the market.

Futures contracts are a highly standardized form of forward contracts:

\begin{itemize}
\tightlist
\item
  that trade on an organized exchange platform (e.g., Chicago Board of Trade, Australian Securities Exchange);
\item
  that specify a product, delivery date, delivery mechanism, and exchange price;
\item
  wherein payments (to and from) are backed by the exchange; and
\item
  wherein the agreement is backed by a good-faith deposit---margin.
\end{itemize}

A futures contract requires two parties---a seller and a buyer. Seller promises to deliver the designated quantity of a commodity in exchange of a fixed price. Selling a contract is also known as taking a \emph{short position}. Buyer promises to receive a delivery of a specified quantity of a commodity for a fixed price. Buying a contract is also known as taking a \emph{long position}.

In practice, very few contracts are held until expiration (when the transfer of ownership would take place). Instead, parties offset their positions. If a trader sells a futures contract, they can offset by buying the same type of futures contract. If a trader buys a futures contract, they can offset by selling the same type of futures contract. When they offset, the contract obligations are fulfilled, and no longer exists the need to accept or make delivery of the product. The only remaining `obligation' for sellers and buyers is any difference in the prices of the two contracts.

\hypertarget{the-relationship-between-spot-and-futures-prices}{%
\section{The Relationship between Spot and Futures Prices}\label{the-relationship-between-spot-and-futures-prices}}

For many agricultural producers, futures markets can be important tools for reducing risk through \emph{hedging}. Hedging implies insuring against the future losses due to price uncertainty by giving up potential future profits.

In practice, hedging means taking the opposite positions in a commodity futures market as compared to what a buyer or seller would take in a spot market in order to guarantee a certain profit. A seller executes a hedge by selling contracts now and buying them later when they sell the commodity in the spot market. A buyer executes a hedge by buying contracts now and selling them later when they purchase the commodity in the spot market.

A it turns out, both for buyers and sellers, the hedge price is equal to the sum of the futures price and the \emph{basis}. Basis refers to the difference between futures and spot prices: \(b_{t} = p_{f,t}-p_{s,t}\). Unlike the futures and spot prices that tend to meander, basis is `stable,' which makes it an attractive measure for empirical analyses.

In discussing how to hedge price risk, we assumed that at the time when a contract expires the futures and spot prices are exactly the same -- i.e., we have, what is known as, the \emph{perfect convergence}. In practice, rarely this is the case, because there are always factors (e.g.~location differentials, storage costs, speculators in the market) that can contribute to disparity in these prices.

\hypertarget{inventories-and-price-of-storage}{%
\section{Inventories and Price of Storage}\label{inventories-and-price-of-storage}}

For an annually produced storable commodity (e.g., grains) supply is given by the production in the current period, plus the beginning inventory, and minus the ending inventory. Decisions about inventories, however, depend on the relationship of the expected price for delivery in the future, relative to the current spot price, and thus both prices are affected by current and expected economic conditions.

Because spot and futures prices are determined simultaneously (and are influenced by the same set of explanatory variables), they are correlated, but differ by a cost of storage. The difference between spot and futures prices is also referred as the price of storage, which is the expected return from holding inventory and hedging. The relationship between the price of storage and the amount of inventory held is illustrated by an upward-sloping supply-of-storage function.

The negative segment of the supply--of--storage curve implies positive inventories even when the return--on--storage is negative. This is due to the convenience yield---a benefit that accrues from having sufficient amount of stocks to maintain continuous business operation or to meet unexpected demands. Marginal convenience yield decreases and approaches zero as stocks increase. Marginal cost of storage remains constant up to a point, but may increase as inventories get large (storage capacity constraints).

\hypertarget{futures-price-responses-to-information}{%
\section{Futures Price Responses to Information}\label{futures-price-responses-to-information}}

Futures markets react to market information and offer best prediction of future spot prices -- the concept that leads to the \emph{efficient market hypothesis}. In a perfect market, the prices for the various futures contracts will exactly anticipate the respective maturity months' spot prices: \[p_{f,t} = E\!\left(p_{s,t+1}|\Omega_t\right)\] That is to say that futures market efficiently aggregates all the information that influences prices, and that futures prices are potentially unbiased forecasts of forthcoming spot prices.

The efficient market hypothesis relies on the following assumptions:
- information is available at no cost;
- current price reflect all known information;
- new information (which occurs randomly) is reflected instantly (and correctly) onto a new price.

The efficient markets support the random walk model, given by: \[p_{f,t+1} = p_{f,t} + \varepsilon_{t+1},\;~~\varepsilon_{t+1}\sim iid\left(0,\sigma^2\right).\]

If markets are inefficient, i.e., if price changes do not fully reflect information changes, then past prices can facilitate economically significant forecasts: \[p_{f,t+1} = \beta_1 p_{f,t} + \beta_2 p_{f,t-1} + \ldots + \varepsilon_{t+1}.\] Not everyone has perfect information (semi-strong-form efficient vs.~strong-form efficient). Even with perfect information, \emph{bounded rationality} may result in suboptimal decisions.

\hypertarget{risk-and-uncertainty}{%
\chapter{Risk and Uncertainty}\label{risk-and-uncertainty}}

Because of complexities of physical and economic systems in agriculture, most processes that economic agents care about unfold with some degree of uncertainty. Uncertainty means that any given action can lead to many possible outcomes, albeit some more likely than others. Thus, decision--making under uncertainty is characterized by risk, as not all possible consequences are equally desirable.

Uncertainty in agricultural markets (or any markets, for that matter) is an issue because a typical person is risk--averse. A person is said to be risk--averse if, for every lottery \(F(w)\), where \(w\) is realized wealth, they will always prefer (at least weakly) the certain amount \(E(w)\) to the lottery itself. That is, \[U\left[\int w dF(w)\right] \ge \int U(w)dF(w).\]

A discrete variant of the foregoing can help with an illustration. A risk--averse person would prefer a guaranteed pay-out of \$100 rather than a pay-out of either \$0 or \$200 each with probability of 50\%.

Under the assumption of monotonically increasing concave utility function, \(U(w)\), we can derive the notions of absolute risk aversion, \[A(w) = -U''(w)/U'(w),\] and of relative risk aversion, \[R(w) = wA(w).\] To the extent that agents will become less averse to a gamble as their wealth grows, \(A(w)\) can be seen as a decreasing function of \(w\); there are no \emph{a priori} expectations regarding any particular behavior of \(R(w)\) with respect to \(w\).

The key sources of uncertainty and risk that face an agricultural producer are related to:

\begin{itemize}
\tightlist
\item
  Production---agricultural producton heavily relies on exogenous factors, such as weather, which can be important source of idiosynchratic uncertainty.
\item
  Price---production decisions are made far in advance of realizing the output and prices.
\item
  Technology---technological advancement may make quasi--fixed investments obsolete.
\item
  Policy---government, through its policies, plays an important role in agriculture, and any changes to these policies creates considerable risk for agricultural investments.
\end{itemize}

\hypertarget{managing-risk-via-futures-markets}{%
\section{Managing Risk via Futures Markets}\label{managing-risk-via-futures-markets}}

One mechanism to mitigate risk in agriculture is via hedging, which shifts the price risk. The acquisition of a futures contract can help manage risk by assuring (approximately) a given level of returns. From the standpoint of an \emph{arbitrage hedge}, if the expected convergence of the spot and futures prices at some future time period is sufficient to cover storage costs, an agent can assure the return by hedging.

Another way to look at hedging on futures markets, is its ability assist a firm in assuring a return from their business activities. A merchant can make a bid on a contract to export certain amount of commodity, without actually having (enough of) it in possession; or, they can offer farmers forward contracts at planting time to purchase their grain at the harvest. This is referred to as an \emph{operational hedge}, which can be seen as a temporary substitute for later transactions in the cash market.

Futures markets can assist a producer to price their output before it is produced. Similarly, a processor can also price the input, prior to the actual purchase of the physical commodity. Thus, an \emph{anticipatory hedge} is based on expected future actions, such as the production and sale of the commodity. An anticipatory hedge can lock in a profit, when price relationships are favourable.

Assuring the return by hedging, crucially relies on the assumption of convergence. Convergence may not be exact (most of the time it is not). As a consequence, hedges do not give perfect results, which is also known as \emph{basis risk}. Thus, basis risk, or our inability to forecast convergence with certainty, poses a problem for hedgers. Another related risk is inherent to the production risk: in times of a negative supply shock, a portion of the hedge, in effect, will turn into a speculative position, which can result in a loss on the position in futures.

\hypertarget{managing-risk-via-insurance}{%
\section{Managing Risk via Insurance}\label{managing-risk-via-insurance}}

An alternative mechanism to mitigate risk in agriculture, which by no means is a substitute to producers' participation in futures markets, is via crop insurance. A substantial production (and price) variability, has led producers realize the importance of risk management as a component of their expected profit-maximization strategies. A producer can take control of risk management, or outsource it. a firm may build up its own funds to cover sudden losses; or a firm can transfer this risk to an insurer for a fee.

Insurance companies are specialized risk bearers that spread the risk over a wide area and groups of people or businesses. Sometimes, government price-- and revenue--supporting programs can serve as a mechanism to mitigate the risk (but availability of such programs vary from country to country). In fact, while crop insurance markets have existed for a long time in some parts of the world, their existence has depended crucially on government support.

Private agricultural insurance markets may fail because the costs of maintaining these markets imply unacceptably low average payouts relative to premiums. Moreover, the perceived demand for crop insurance may be overstated because farmers have an option to manage the risk individually.

General features of insurance programs are: they are based on historical data. they can be farm-based or area-based. indemnity is triggered if the revenue falls below some threshold. a producer needs to pay fee, premium, to participate in the program. the government may choose to subsidize the program.

\hypertarget{insurance-market-failures}{%
\section{Insurance Market Failures}\label{insurance-market-failures}}

Two main issues that are associated with crop insurance (or any insurance for that matter) are \emph{moral hazard} and \emph{adverse selection}. Moral hazard implies that a farmer's optimal decision may be different with insurance as compared to that without insurance. Adverse selection implies that farmers that choose to insure at a given rate, are also the ones who are more exposed to risk.

\hypertarget{moral-hazard}{%
\subsection{Moral Hazard}\label{moral-hazard}}

Moral hazard problems arise, when the insurer contracts on a risk-averse producer whose inputs are unobservable. With or without moral hazard, the contract is designed in such way that a farmer is willing to participate in the program. Under moral hazard, however, it is no longer optimal for the risk-neutral insurer (principal) to assume all risk; some residual risk must be borne by the risk-averse producer (agent).

The implications of the moral hazard problem are not as clear-cut as intuition might suggest. On the one hand, being relieved of some of the consequences of low input use, the producer may reduce input intensity. On the other hand, if input use is risk-increasing, then a high-risk environment may cause the producer to use fewer inputs than a low-risk environment.

\hypertarget{adverse-selection}{%
\subsection{Adverse Selection}\label{adverse-selection}}

When the insurer is not completely informed about the nature of the risk being insured, then the insurer faces the problem of adverse selection. Avoiding adverse selection may require the successful crop insurance program to identify, acquire, and skilfully use data that discriminate among different risks. While costly to implement, such data management procedures may be crucial because, unless rates are perceived as being acceptable, the market may collapse.

Identifying a sufficiently large number of relatively homogeneous risks is a prerequisite for a successful contract. Another factor that can facilitate to sustain the contract is sufficiently high degree of risk--aversion among producers.

\hypertarget{mitigating-the-market-failures}{%
\section{Mitigating the Market Failures}\label{mitigating-the-market-failures}}

While moral hazard and adverse selection are conceptually distinct phenomena, they share information asymmetry as a common characteristic feature. To that end, these two issues can be interrelated, as illustrated in the following example: Consider a wheat and corn farmer who has one hectare of high quality land and one hectare of low quality land. Given the decision to insure wheat but not corn, the planting of wheat on poor quality land is moral hazard. Given the decision to plant wheat on poor quality land, the insuring wheat is adverse selection.

Due to the informational nature of main barriers to successful crop insurance markets, the obvious solution is to facilitate better (ideally, complete) information. One approach to reduce adverse selection is to obtain and use farm-level (rather than area-level) information; but this creates incentives for moral hazard. Alternatively, area yield insurance can solve moral hazard and possibly mitigate adverse selection, but this will not completely alleviate the production risk.

Revenue insurance, instead of yield insurance, mitigates somewhat the incidence of moral hazard and adverse selection. It also serves the main purpose of crop insurance as it addresses the issue of income risk facing producers.

Yet another crop insurance mechanism, that can mitigate some of the aforementioned issues, is the so-called \emph{weather derivatives}. Such insurance is applicable because weather is intrinsically linked with agricultural production. Weather derivatives are based on historical weather data of the area where farm is located. Indemnity is triggered if weather conditions are ``bad'' relative to the historical normal. But correlation (or the lack of it) with yields can be an issue (e.g., not al farms are located in close proximity to weather stations).

\hypertarget{agricultural-policy}{%
\chapter{Agricultural Policy}\label{agricultural-policy}}

\hypertarget{references}{%
\chapter*{References}\label{references}}
\addcontentsline{toc}{chapter}{References}

\hypertarget{refs}{}
\leavevmode\hypertarget{ref-barrett2002}{}%
Barrett, Christopher B, and Jau Rong Li. 2002. ``Distinguishing Between Equilibrium and Integration in Spatial Price Analysis.'' \emph{American Journal of Agricultural Economics} 84 (2): 292--307.

\leavevmode\hypertarget{ref-bellemare2018}{}%
Bellemare, Marc F, and Jeffrey R Bloem. 2018. ``Does Contract Farming Improve Welfare? A Review.'' \emph{World Development} 112: 259--71.

\leavevmode\hypertarget{ref-goodwin2002}{}%
Goodwin, Barry K, Thomas J Grennes, and Lee A Craig. 2002. ``Mechanical Refrigeration and the Integration of Perishable Commodity Markets.'' \emph{Explorations in Economic History} 39 (2): 154--82.

\leavevmode\hypertarget{ref-grant2012}{}%
Grant, Jason H, and Kathryn A Boys. 2012. ``Agricultural Trade and the GATT/WTO: Does Membership Make a Difference?'' \emph{American Journal of Agricultural Economics} 94 (1): 1--24.

\leavevmode\hypertarget{ref-myers2010}{}%
Myers, Robert J, Richard J Sexton, and William G Tomek. 2010. ``A Century of Research on Agricultural Markets.'' \emph{American Journal of Agricultural Economics} 92 (2): 376--403.

\leavevmode\hypertarget{ref-tomek2014}{}%
Tomek, William G, and Harry M Kaiser. 2014. \emph{Agricultural Product Prices}. 5th ed. Cornell University Press.

\end{document}
